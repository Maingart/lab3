\def \equationFirst {\sin^2{y'} - \cos{y} = e^{xy} \cdot \cos^2{y'}}
\def \conditionFirst { y(1) = -2\pi }
\def \toFindFirst { y(\pi) }

\def \equationSecond {\tan{(xy^2)} - \dfrac{5e^{4x}}{y} = \arcsin{y'} + 1}
\def \conditionSecond { y(0) = -2 }
\def \toFindSecond { y\left(\dfrac{\pi}{3}\right) }

\input{common/cpprules.tex}

\begin{scontents}
\begin{lstlisting}
#include <iostream>
#include <fstream>
#include <cmath> 

typedef double F(double,double);

double euler(F f, double y0, double x0, double xN, double h)
{
	std::ofstream fout("ode_results1.txt"); 
	double y = y0;
	for (double x = x0; x < xN; x += h) {
		y += h * f(x, y);
		fout << x << ' ' << y << '\n';
	}
	fout.close();
	return y;
}

double fun(double x, double y) {
	return acos(sqrt((1 - cos(y)) / (exp(x*y) + 1)));
}

int main()
{
	euler(fun, -2*M_PI, 1, M_PI, 0.00001);			  
}

\end{lstlisting}
\end{scontents}

\begin{scontents}
\begin{lstlisting}
#include <iostream>
#include <fstream>
#include <cmath> 

typedef double F(double,double);

double euler(F f, double y0, double x0, double xN, double h)
{
	std::ofstream fout("ode_results2.txt"); 
	double y = y0;
	for (double x = x0; x < xN; x += h) {
		y += h * f(x, y);
		fout << x << ' ' << y << '\n';
	}
	fout.close();
	return y;
}

double fun(double x, double y) {
	return sin(tan(x * y * y) - 5 * exp(4 * x) / y - 1);
}

int main()
{
	euler(fun, -2, 0, M_PI/3, 0.00001);
}

\end{lstlisting}
\end{scontents}

\newcommand{\solutionItemSecond}[5]{
	\item
	    \customCases{#2}{#3}
	    
	    
	    \vspace{1em}
	    \textit{Решим численно:} $ #4 \approx #5 $
	    
 	    \vspace{1em}
	    \textit{График:}
	    \begin{figure}[h]
	    	\centering
	    	\includegraphics[width=\textwidth]{img/euler#1.pdf}
	    \end{figure}
    
    	\clearpage
    	\textit{Код:}
    		\getstored[#1]{contents}
}

