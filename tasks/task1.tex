\def \equationFirst {\dfrac{y}{y'^2} = 1 - \tan{y'}}

\def \equationSecond {{y'}^2 - 2xy' = 8x^2}

\def \equationThird {{y'^2 - yy' + 4e^x = 0}}

\def \equationFourth {y = \dfrac{3}{2}xy' + e^{y'}}

\def \equationFifth {\ln{(xy'^2 - yy')} = y'\ln{\sin{y'} + \ln{y'}}}

\newcommand{\solutionItemFirst}[3]{
    \item
        $ #1 $

		\textit{Характеристика:} 
		\begin{tabular}[t]{p{9cm}} 
			#2
		\end{tabular}

        \textit{Общее решение:} #3
}


\section{Задача 1. Дать характеристику}
\subsection{Постановка задачи}
Для следующих дифференциальных уравнений дать характеристику и найти
общее решение с помощью программ компьютерной математики:

\begin{enumerate}
    \item $ \equationFirst $
    \item $ \equationSecond $
    \item $ \equationThird $
    \item $ \equationFourth $
    \item $ \equationFifth $
\end{enumerate}

\subsection{Решение}
\begin{enumerate}
    \solutionItemFirst
        {\equationFirst}
        {
        	Уравнение, неразрешенное относительно производной,
        	вида $ F(y, y') = 0 $. \\
        	Разрешимо относительно искомой функции $ y $.
        }
        {
        	\customCases
        		{y = p^2 - p^2 \tan{p}}
        		{x = \ln{\cos{p}} - p(\tan{p} + 2) + C}
        }

	\clearpage

    \solutionItemFirst
	    {\equationSecond}
	    {
	    	Уравнение, неразрешенное относительно производной,
	    	вида $ F(x, y') = 0 $. \\
	    	Разрешимо, как квадратное уравнение относительно $ y' $.
    	}
	    {
	    	\customCases
				{y = C_1 - x^2}
				{y = 2x^2 + C_2}
    	}

	\vspace{2em}

	\solutionItemFirst
		{\equationThird}
		{
			Уравнение, неразрешенное относительно производной,
			вида $ F(x, y, y') = 0 $. \\
			Разрешимо, как квадратное уравнение относительно $ y' $.
		}
		{
			\customCases
				{{y} = C_1 - x^2}
				{{y} = 2x^2 + C_2}
		}

	\vspace{2em}

	\solutionItemFirst
		{\equationFourth}
		{
			Уравнение, неразрешенное относительно производной,
			вида $ F(x, y, y') = 0 $. \\
			Разрешимо, как уравнение Лагранжа.
		}
		{
			\customCases
				{x=\dfrac{C-(2p^2-4p+4)e^p}{p^3}}	
				{y=\dfrac{3C-2(3p^2-6p+6)e^p}{2p^2} + e^p}
		}

	\clearpage

	\solutionItemFirst
		{\equationFifth}
		{
			Уравнение, неразрешенное относительно производной,
			вида $ F(x, y, y') = 0 $. \\
			Сводится к уравнению Клеро.
		}
		{$ y=Cx-e^C\sin{C} $}

\end{enumerate} 
